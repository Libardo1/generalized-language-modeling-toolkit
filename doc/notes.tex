\documentclass[11pt,a4paper]{article}

\usepackage{mathtools}
\usepackage{amssymb}

\usepackage[parfill]{parskip}

\usepackage{hyperref}

\newcommand{\notimplies}{%
  \mathrel{{\ooalign{\hidewidth$\not\phantom{=}$\hidewidth\cr$\implies$}}}}

\newcommand{\Seq}{\mathcal{S}}
\newcommand{\Hist}{\mathcal{H}}
\newcommand{\SeqF}{\mathcal{S}_F}
\newcommand{\HistF}{\mathcal{H}_F}
\newcommand{\Skp}{\rule{7pt}{.5pt}}

\title{GLMTK --- Notes}
\author{Lukas Schmelzeisen \\ \texttt{lukas@uni-koblenz.de}}
\date{\today}

\begin{document}
  \maketitle
  \tableofcontents
  \clearpage

  \section{Notation}

  \begin{tabular}{ l l }
    $\Sigma^n$  & Set of all $n$-Grams \\
    $\Skp$ & Skipped Word \\
    $\Seq_i$ & $i$-th Word in Sequence \\
    $|\Seq|$ & Number of Words in Sequence \\
    $c(\Seq)$ & Absolute Count of Sequence \\
    $N_{1+}(\Seq)$ & Continuation Count of Sequence \\
    $N = c(\Skp)$ & Number of Words \\
    $V = N_{1+}(\Skp)$ & Vocabulary Size \\
  \end{tabular}

  \section{$n$-Gram Probability Estimators}

  A $n$-Gram Probability Estimator is a function $P :\Sigma^n \to [0,1]$ which
  returns the probability of a $n$-Gram \emph{Sequence} $\Seq$ for a fixed
  $n$-Gram \emph{History} $\Hist$.

  For easier handling we define the \emph{Full Sequence} as the concatenation
  of history and sequence ($\SeqF = \Hist * \Seq$), and the \emph{Full History}
  as the concatenation of history and skipped sequence
  ($\HistF = \Hist * \underbrace{~ \Skp \dotso ~}_{\mathclap{|\Seq| ~ \text{many}}}$)

  An observation on the counts of $n$-Grams:
  \begin{equation}
    c(\Hist) = 0 \implies c(\HistF) = 0 \implies c(\SeqF) = 0
  \end{equation}

  For histories we define the predicate of a (un-)seen history. Note that this
  defines the empty history as ``seen'', which is a choice that was made in
  order to make the definitions and implementations of estimators more natural.
  \begin{equation}
    \begin{aligned}
      \Hist ~ \mathrm{seen} &\iff \Hist = \varnothing \lor c(\Hist) \neq 0 \\
      \Hist ~ \mathrm{unseen} &\iff \Hist \neq \varnothing \land c(\Hist) = 0
    \end{aligned}
  \end{equation}

  \subsection{Kinds of Probability Estimators}

  $n$-Gram probability estimators can be separated into two categories,
  according to which mathematical type of probability they implement:
  \emph{Conditional Probabilities} or \emph{Marginal Probabilities}. How you
  actually estimate the probability of a sequence depends on what kind of
  estimator you are using.

  $n$-Gram probability with conditional probability estimators:
  \begin{equation}
    P(w_1^n) = P(w_n | w_1^{n-1}) \cdot P(w_{n-1} ~ \Skp ~ | w_1^{n-2}) \dotsm P(w_1 ~ \Skp ~ \Skp ~ \dotso)
  \end{equation}

  $n$-Gram probability with marginal probability estimators:
  \begin{equation}
    P(w_1^n) = P(w_n | w_1^{n-1}) \cdot P(w_{n-1} | w_1^{n-2}) \dotsm P(w_1)
  \end{equation}

  Conditional and marginal probabilities differ  on how they handle the case of
  an unseen history. Conditional probabilities have
  $P(\Seq | \Hist ~ \mathrm{unseen}) = 0$
  while marginal probabilities have
  $P(\Seq | \Hist ~ \mathrm{unseen}) = P_\mathrm{Substitute}(\Seq | \Hist)$.
  % TODO: Check if this statement is true for unseen histories or just for P(H) = 0.

  \subsection{Tests for Probability Estimators}

  In order for probability estimators to be probability measures, the
  following equations / tests should hold:

  \texttt{NGramProbabilitiesSumTest}:
  \begin{equation}
    \sum_{\Seq \in \Sigma^n} P(\Seq) = 1
  \end{equation}

  \texttt{FixedHistoryProbabilitiesSumTest}:
  \begin{alignat}{2}
    \intertext{Conditional:}
    \forall \Hist \in \Sigma^n :
    ~&(\Hist ~ \Skp ~ \mathrm{seen} &&\implies \sum_{\Seq \in \Sigma} P(\Seq | \Hist) = 1)  ~ \land \\
    ~&(\Hist ~ \Skp ~ \mathrm{unseen} &&\implies \sum_{\Seq \in \Sigma} P(\Seq | \Hist) = 0) \notag \\
    \intertext{Marginal:}
    \forall \Hist \in \Sigma^n:
    ~&\sum_{\Seq \in \Sigma} P(\Seq | \Hist&&) = 1
  \end{alignat}

  \subsection{Substitute Probability Estimators}

  Substitute Probability Estimators are used in a context where other
  probability estimators cannot use their usual algorithm to estimate the
  probability of a sequence. They then instead use $P_\mathrm{Substitute}$ to
  calculate that probability.

  Let $P_\mathrm{Substitute} \in \big\{ P_\mathrm{Uniform} , P_\mathrm{AbsUnigram} , P_\mathrm{ContUnigram}\big\}$
  fixed globally at program start.
  \begin{align}
    P_\mathrm{Uniform}(\Seq | \Hist) &= \frac{1}{V} \\
    P_\mathrm{AbsUnigram}(\Seq | \Hist) &= \frac{c\big(\Seq_1\Big)}{N} \\
    P_\mathrm{ContUnigram}(\Seq | \Hist) &= \frac{N_{1+}\big(\Skp ~ \Seq_1\big)}{N_{1+}\big(\Skp ~ \Skp\big)}
  \end{align}

  All substitute probability estimators are marginal probability estimators.

  \subsection{Fraction Estimators}

  Fraction Estimators are probability estimators that have the form $\frac{n}{d}$.
  \begin{alignat}{2}
    \intertext{Conditional:}
    P_\mathrm{Frac}{\scriptstyle[n , d]}(\Seq | \Hist) &= \begin{cases}
      0\hphantom{_\mathrm{Substitute}(\Seq | \Hist)} & \text{if} ~ \Hist ~ \mathrm{unseen} \lor d = 0 \\
      \frac{n}{d} & \text{else}
    \end{cases} \\
    \intertext{Marginal:}
    P_\mathrm{Frac}{\scriptstyle[n , d]}(\Seq | \Hist) &= \begin{cases}
      P_\mathrm{Substitute}(\Seq | \Hist) & \text{if} ~ \Hist ~ \mathrm{unseen} \lor d = 0 \\
      \frac{n}{d} & \text{else}
    \end{cases}
  \end{alignat}

  \subsubsection{MaximumLikelihoodEstimator}

  \begin{equation}
    P_\mathrm{MLE}(\Seq | \Hist) = P_\mathrm{Frac}{\scriptstyle[c(\SeqF) , c(\HistF)]}(\Seq | \Hist)
  \end{equation}
  % TODO: If history is empty we use c(\Skp) instead of c(\HistF).

  \subsubsection{``False''MaximumLikelihoodEstimator}
  % TODO: Find better name.

  FMLE only works in the marginal probability setting.

  \begin{equation}
    P_\mathrm{FMLE}(\Seq | \Hist) = P_\mathrm{Frac}{\scriptstyle[c(\SeqF) , c(\Hist)]}(\Seq | \Hist)
  \end{equation}
  % TODO: If history is empty we use c(\Skp) instead of c(\Hist).

  \subsubsection{ContinuationMaximumLikelihoodEstimator}

  \begin{equation}
    P_\mathrm{CMLE}(\Seq | \Hist) = P_\mathrm{Frac}{\scriptstyle[N_{1+}(\Skp ~ \SeqF) , N_{1+}(\Skp ~ \HistF)]}(\Seq | \Hist)
  \end{equation}

  % TODO: FixedHistoryProbabilitiesSumTest has _ H _ check.

  \subsection{DiscountEstimator}

  A Discount Estimator takes any kind of fraction estimator and subtracts some
  discount from the numerator, in order to free probability mass to be used for
  smoothing. Obviously this means, that discount estimators are no longer
  probability estimators, and will not pass tests. Instead they have to be used
  in conjunction with an Interpolation Estimator.

  \begin{equation}
    P_\mathrm{Discount}{\scriptstyle[D,P_\mathrm{Frac}[n,d]]}(\Seq | \Hist) = P_\mathrm{Frac}{\scriptstyle[\max(0,n-D),d]}(\Seq | \Hist)
  \end{equation}
  With $D: \Hist \to \mathbb{R}$ and $D(\Hist)<1 ~ \forall \Hist$.

  \subsubsection{AbsoluteDiscountEstimator}

  \begin{equation}
    P_\mathrm{AbsDiscount}{\scriptstyle[D,P_\mathrm{Frac}[n,d]]}(\Seq | \Hist) = P_\mathrm{Discount}{\scriptstyle[D,P_\mathrm{Frac}[n,d]]}(\Seq | \Hist)
  \end{equation}
  With $D \in \mathbb{R}$ and $D < 1$.

\end{document}
